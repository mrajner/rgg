\documentclass{rgg}

%% this package is included in this example only to list
%% the source code at the end of this example
%% the authors can remove this declaration
\usepackage{verbatim}

\title[The sample paper\ldots]{%
  The sample paper and guidelines for using \LaTeX{} class for submission to 
  Reports on Geodesy and Geoinformatics}
\author[Smith, J., White, J.]{John Smith$^{1,2}$, Jerry White$^2$}
\affiliation{%
  $^1$Department of Very Important Issues,\\
  Faculty of Even More Important Issues,\\
  Awesome University
  \nextaffiliation
  $^2$Department of Geodesy Affairs,\\
  Institute of Geodesy and Geoinformatcis
}

\begin{document}
    \maketitle

    \begin{abstract}
      These paper give you template for preparing article for ``Reports on 
      Geodesy and Geoinformatics''. The authors are requested to prepare the 
      paper according to the scheme given below. The length of the abstract 
      should be not larger than 200\,–\,250 words.
      \keywords{insert 3\,–\,4 keywords or phrases}
    \end{abstract}

  \section{General information}
    Use \texttt{rgg} \LaTeX{} class for article preparation.
    Make sure that \texttt{rgg.cls} is assesible in your
    \texttt{TEXPATH} variable or this class file can be put
    in your working directory with other \texttt{.tex} file(s).

    Using external packages and user defined commands is allowed
    but do not change the formatting settings.
    Remember to send all your files (except \texttt{rgg.cls}) 
    with all your settings and inputs.

  \section{Main structure}
    \label{labelforsection}

    Following sections gives the basic rules for formatting your
    paper.

  \subsection{Title, author info and abstract}
    Use \texttt{title}, \texttt{author} 
    and \texttt{affiliation} command
    to define your title page. This can be done
    in your preamble or at the beginning of the document.
    To make titling visible use \texttt{maketitle} command.

    Optionally specify short version of author(s) and title information
    for running header (again example within this file in the square brackets).

    Put appropriate reference number near each author in superscript
    and refer this number in special \texttt{affiliation} command.
    If all authors has the same affiliation you can skip superscripts.
    Put your affiliation info in \texttt{affiliation} command
    and separate several affiliations with \texttt{nextaffiliation} command.

    Please put short note about author(s)
    in \texttt{authornote} command 
    at the end of article with short information about
    contributors. Put your electronic address within
    \texttt{email} command.
    The electronic mail address of all Authors is indispensable.

  \subsection{Sectioning}
    \label{labelforsubsection}
    For sectioning use standard \LaTeX{} commands
    \texttt{section} and \texttt{subsection}.
    All numbering will be added automatically.
    To refer to section or subsection number
    simply use \texttt{label} just after the section
    name and then you can freely refer to it in subsequent text
    using \texttt{ref} and label name.
    Just like here for section~\ref{labelforsection} or
    subsection~\ref{labelforsubsection}.
    Do not go to deep in sectioning, \emph{i.e.} try to avoid
    \texttt{subsubsection}s. 

  \subsection{Figures}
    You should not worry about figure placement.
    Just wrap your graphics in \texttt{figure}
    environment and use standard \texttt{caption}
    command as in example below.
    To refer to this figure again use pair of \texttt{label}
    \texttt{ref} commands.
    \begin{figure}
      \includegraphics{figure}
      \caption{This is caption of Figure~\ref{figurecaption}}
      \label{figurecaption}
    \end{figure}

  \subsection{Tables}
    Put your table definition in 
    \texttt{table} environment and give appropriate caption.
    All formatting and numbering will be set automatically.
    The \texttt{tabular} environment is wrapped into 
    \texttt{table} environment hence it is floating object.
    If your table needs to be putted at specific place
    do not omit \texttt{table} environment but put
    optional \texttt{[H]} specifier
    (see tables~\ref{labelforheretable} and \ref{labelforgloatingtable}).

    \begin{table}[H]
      \caption{Sample table which shows within text}
      \label{labelforheretable}
      \begin{tabular}{ccc}
        \toprule
        header1 & header2 & header3 \\
        \midrule
        first row  & second column & last column \\
        \bottomrule
      \end{tabular}
    \end{table}

    \begin{table}[b]
      \caption{Sample floating table}
      \label{labelforgloatingtable}
      \begin{tabular}{@{}lll@{}}
        \toprule
        header1 & header2 & header3 \\
        \midrule
        first row  & second column & last column \\
        \bottomrule
      \end{tabular}
    \end{table}

  \subsection{Equations}
    Using and referencing equations 
    is given with standard \LaTeX{} macros.
    \begin{equation}
      \frac{1}{m\cdot c^2} = \frac{1}{E}
      \label{equationlabel}
    \end{equation}
    This equation will be numbered properly and you
    can refer to them with appropriate
    \texttt{label} and \texttt{ref} command.
    Just like here in Equation~\ref{equationlabel}.

  \subsection{References}
    You can still make your citation by hand writing, for example
    \begin{quotation}
      Generalization of least-squares was given in literature (Neitzel, 2010)
    \end{quotation}
    or 
    \begin{quotation}
      Newton (1687) said that the Earth is flattened at the poles. 
    \end{quotation}
    Within this method you have to replace \texttt{bibliography}
    command with unnumbered section command (\texttt{section*\{References\}})
    and then put all your bibliography there. Make sure to format bibliography
    according to \texttt{apa} rules.

    The preferred and recommended method here is to use \texttt{bibtex}
    tool. Within this method all formatting will be done automatically.
    See the attached \texttt{rgg\_sample\_article.bib} file.
    Make sure to run \texttt{bibtex} command after compiling you source
    and afterwards compile in two times more to resolve appropriately all cross 
    references.
    With this method citing should be done with commands given
    in \texttt{apacite} package documentation.
    To give a~very quick introduction we list basic usage of these commands in 
    Table~\ref{apacitecommands}.
    \begin{table}
      \caption{Citing with \texttt{apacite} package}
      \label{apacitecommands}
      \begin{tabular}{ll}
        \toprule
        command                                  & output                            \\
        \midrule
        \verb|\cite{Neitzel2010}|                & \cite{Neitzel2010}                \\
        \verb|\citeA{Neitzel2010,mastersthesis}| & \citeA{Neitzel2010,mastersthesis} \\
        \verb|\cite{book}|                       & \cite{book}                       \\
        \verb|\citeNP{phdthesis}|                & \citeNP{phdthesis}                \\
        \bottomrule
      \end{tabular}
    \end{table}

  \section{Closing remarks}
    The Author(s) are kindly requested to submit, with the manuscript, 
    the names, and e-mail addresses of five potential reviewers. Note 
    that the editor retains the sole right to decide whether or not the 
    suggested reviewers are used. The names of potential reviewers should 
    be entered in the section Comments for the Editor in the 1St step of 
    on-line submission process.

    The \texttt{rgg.cls} use some standard packages.
    All modern distribution of \LaTeX{} compilers should include
    these external macros.
    If you encounter any problems read log file and if any dependencies are 
    missing you still can download it from \TeX{} repositories.

    Send all your external files (graphisc, bibliography settings) while
    submitting to Reports on Geodesy and Geoinformatics.
    The editors reserve the right to modify your source files to adapt it to 
    journal requirements.

    \acknowledgement{
      If you would like to use any acknowledgement 
      please insert it into
      \texttt{acknowledgement} command.
    }

    \bibliography{rgg_sample_article.bib}

    \authornote{%
      Professor John Smith$^{1,2}$ \email{johnsmith@xxx.com}\\
      PhD Jerry White$^2$ \email{jerry@yyy.pl}\\
      $^1$Department of Very Important Issues

      Faculty of Even More Important Issues,\\
      Awesome University\\
      Awesomness St. 8, 00-000, Freedomtown, Freedom Country

      $^2$Department of Geodesy Affairs,\\
      Institute of Geodesy and Geoinformatcis\\
      North Pole Street, South Pole
    }

    %% this last part is just inclusion of source file 
    %% only for documentation purposes
    \clearpage
  \section*{Appendix}

  \subsection*{Listing of \texttt{rgg\_sample\_article.tex} file
      (source of this document)}
    {\scriptsize\verbatiminput{./rgg_sample_article.tex}}

  \subsection*{Listing of \texttt{rgg\_sample\_article.bib} file 
      (source of bibliography)}
    {\scriptsize\verbatiminput{rgg_sample_article.bib}}

\end{document}

