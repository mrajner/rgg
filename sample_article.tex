\documentclass{rog}

% 3. For the title and subtitles use type size 12 with Arial bold font.
% 5. Capitalize the first word and all other principal words of the titles and subtitles.
% 6. The titles should be centred, the subtitles initial caps left, numbering of subsections by decimal system, e.g. 1.1, 1.2, 1.3 etc.
% 7. The title should be followed by name(s) of the Author(s) and his/their affiliation (all centred).
% 8. Underneath the authors' affiliation insert one blank line and write abstract (type size 11 with Arial Italic font).
% 9. Underneath the abstract add one blank line and write Keywords (type size 12 with Arial font).
% 10. The figures and subscriptions should be numbered below the figures and centred, e.g. >"Fig. 3. Text"< (align to the left of figure, use the font size of 11 points).
% 11. The tables and the attached text should be numbered on the top of the table and centred, e.g. >"Table 2. Text< (align to the left of table, use the font size of 11 points).
% 12. Margins left, right, top, bottom: 2.5 cm; for the first page containing title top margin 5 cm.

% 13. Between sections of the text add one empty line.

% 14. Use the APA format as a style of citation and reference list (see Guidelines for APA Citation Style with examples)

% 15. The address for correspondence, e.g. name with title, address of the institution,should be included at the end of the article. The electronic mail address of all Authors is indispensable for sending a proof and final pdf file.
% 16. The paper (with main text, abstract and keywords) should be submitted via web page.

\title{The sample paper and guidelines for using \LaTeX{} class for submission to Reports on Geodesy and Geoinformatics}
\author[Smith, J., White, J.]{John Smith$^1$, Jerry White$^2$}
\affiliation{1}

\begin{document}

    \maketitle

    \begin{abstract}
      These paper give you template for preparing article for ``Reports on 
      Geodesy and Geoinformatics''. The authors are requested to prepare the 
      paper according to the scheme given below. The length of the abstract 
      should be not larger than 200\,–\,250 words.
      \keywords{insert 3 – 4 keywords or phrases}
    \end{abstract}


  \section{Main structure}
    \label{labelforsection}
    For sectioning use standard \LaTeX{} commands
    \verb$\section$ and \verb$\subsection$.
    All numbering will be added automatically.
    To refer to section or subsection number
    simply use \verb$\label$ just after the section
    name and then you can freely refer to it in subsequent text
    using \verb$\ref$ and label name.
    Just like here for section~\ref{labelforsection} or
    subsection~\ref{labelforsubsection}.

  \subsection{Subsectioning}
    \label{labelforsubsection}
    This dummy subsection shows how to use subsections in document.

  \section{figures}
    You should not worry about figure placement.
    Just wrap your graphics in \texttt{figure}
    environment.
  \begin{figure}
      \includegraphics[width=0.5\textwidth]{figure}
      \caption{This is caption of figure}
    \end{figure}

  \section{tables}
    \begin{table}
      \caption{Sample table}
    \begin{tabular}{ccr}
      \toprule
      header1 & header2 & header3 \\
      \midrule
      first row & second column & last column\\
      \bottomrule
    \end{tabular}
  \end{table}

  \section{Equations}
    Using and referencing equations 
    is given with standard \LaTeX{} macros.
    To use
    \begin{equation}
      dfsf
    \end{equation}

    
  \section{Closing remarks}
    The Author(s) are kindly requested to submit, with the manuscript, 
    the names, and e-mail addresses of five potential reviewers. Note 
    that the editor retains the sole right to decide whether or not the 
    suggested reviewers are used. The names of potential reviewers should 
    be entered in the section Comments for the Editor in the 1st step of 
    online submission process.

    \acknowledgement{
      Please insert your acknowledgement in 
      \verb$\acknowledgement$ command 
    }

\end{document}

The figures and subscriptions should be numbered below the figures and centred, 
e.g. >"Fig. 1. Text"< (align to the left of figure, use the font size of 11 
  points). 



Fig. 1. Title of figure

For equations it is recommended to use standard equation editor existing in Word editor. We accept insertion of the equations into tables. Insert the equation number on the right side. See example:


(1)

The tables should be numbered on the top of the table and centred, e.g. >"Table 1. Text< (align to the left of table, use the font size of 11 points).

Table 1. Title of table


































These paper presents example for preparing article for “Reports on 
Geodesy and Geoinformatics”. We require APA Citation Style. You can find 

Acknowledgement

References 

Neitzel, F. (2010). Generalization of total least-squares on example of unweighted and weighted 2D similarity transformation. Journal of Geodesy, 84(12), 751-762. doi:10.1007/s00190-010-0408-0

Teunissen, P. J. G., & de Bakker, P. F. (2013). Single-receiver single-channel multi-frequency GNSS integrity: outliers, slips, and ionospheric disturbances. Journal of Geodesy, 87(2), 161-177. doi:10.1007/s00190-012-0588-x

Knight, N. L., Wang, J., & Rizos, C. (2010). Generalised measures of reliability for multiple outliers. Journal of Geodesy, 84(10), 625-635. doi:10.1007/s00190-010-0392-4


Authors: Short information about authors: name with title, address of the institution, email address.

