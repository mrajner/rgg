\documentclass{rgg}
\flushbottom

%% this package is included in this example only to list
%% the source code at the end of this example
%% the authors can remove this declaration
\usepackage{verbatim}

% 3. For the title and subtitles use type size 12 with Arial bold font.
% 5. Capitalize the first word and all other principal words of the titles and subtitles.
% 6. The titles should be centred, the subtitles initial caps left, numbering of subsections by decimal system, e.g. 1.1, 1.2, 1.3 etc.
% 7. The title should be followed by name(s) of the Author(s) and his/their affiliation (all centred).
% 8. Underneath the authors' affiliation insert one blank line and write abstract (type size 11 with Arial Italic font).
% 9. Underneath the abstract add one blank line and write Keywords (type size 12 with Arial font).
% 10. The figures and subscriptions should be numbered below the figures and centred, e.g. >"Fig. 3. Text"< (align to the left of figure, use the font size of 11 points).
% 11. The tables and the attached text should be numbered on the top of the table and centred, e.g. >"Table 2. Text< (align to the left of table, use the font size of 11 points).
% 12. Margins left, right, top, bottom: 2.5 cm; for the first page containing title top margin 5 cm.

% 13. Between sections of the text add one empty line.

% 14. Use the APA format as a style of citation and reference list (see Guidelines for APA Citation Style with examples)

% 15. The address for correspondence, e.g. name with title, address of the institution,should be included at the end of the article. The electronic mail address of all Authors is indispensable for sending a proof and final pdf file.
% 16. The paper (with main text, abstract and keywords) should be submitted via web page.

\title[The sample paper\ldots]{The sample paper and guidelines for using \LaTeX{} class for submission to Reports on Geodesy and Geoinformatics}
\author[Smith, J., White, J.]{John Smith$^{1,2}$, Jerry White$^2$}
\affiliation{1 Department of Very Important Issues}


\begin{document}

    \maketitle

    \begin{abstract}
      These paper give you template for preparing article for ``Reports on 
      Geodesy and Geoinformatics''. The authors are requested to prepare the 
      paper according to the scheme given below. The length of the abstract 
      should be not larger than 200\,–\,250 words.
      \keywords{insert 3 – 4 keywords or phrases}
    \end{abstract}

  \section{General information}
    Use \texttt{rgg} LaTeX class for article preparation.
    Make sure that \texttt{rgg.cls} is assesible in your
    \texttt{TEXPATH} variable or this class file can be put
    in your working directory with other \texttt{.tex} file(s).

    Using external packages and user defined commands is allowed
    but do not change the formatting settings.
    Remember to send all your files (except \texttt{rgg.cls}) 
    with all your settings and inputs.
    
    
  \section{Main structure}
    \label{labelforsection}

    Following sections gives the basic rules for formatting your
    paper.

  \subsection{Title, author info and abstract}
    Use \texttt{title}, \texttt{author} 
    and \texttt{affiliation} command
    to define your title page. This can be done
    in your preamble or at the beginning of the document.
    To make titling visible use \texttt{maketitle} command.

    Optionally specify short version of author(s) and title information
    for running header (again example within this file in the square brackets).

    Put appropriate reference number near each author in superscript
    and refer this number in special \texttt{affiliation} command.
    If all authors has the same affiliation you can skip superscripts.
    
    Please put short note about author(s)
    in \texttt{authornote} command 
    at the end of article with short information about
    contributors.


  \subsection{Sectioning}
    \label{labelforsubsection}
    For sectioning use standard \LaTeX{} commands
    \texttt{section} and \texttt{subsection}.
    All numbering will be added automatically.
    To refer to section or subsection number
    simply use \texttt{label} just after the section
    name and then you can freely refer to it in subsequent text
    using \texttt{ref} and label name.
    Just like here for section~\ref{labelforsection} or
    subsection~\ref{labelforsubsection}.
    Do not go to deep \emph{i.e.} try to avoid
    \texttt{subsubsection}s. 


  \subsection{Figures}
    You should not worry about figure placement.
    Just wrap your graphics in \texttt{figure}
    environment and use standard \texttt{caption}
    command as in example below.
    To refer to this figure
    \begin{figure}
      \includegraphics{figure}
      \caption{This is caption of Figure~\ref{figurecaption}}
      \label{figurecaption}
    \end{figure}

  \subsection{Tables}
    Put your table definition in 
    \texttt{table} environment and give appropriate caption.
    All formatting and numbering will be set automatically.
    The \texttt{tabular} environment is wrapped into 
    \texttt{table} environment hence it is floating object.
    If your table needs to be putted at specific place
    do not omit \texttt{table} environment but put
    optional \texttt{[H]} specifier.

    \begin{table}[H]
      \caption{Sample table which shows within text}
      \label{labelforheretable}
      \begin{tabular}{ccc}
        \toprule
        header1 & header2 & header3 \\
        \midrule
        first row  & second column & last column \\
        second row & second column & last column \\
        \bottomrule
      \end{tabular}
    \end{table}

    \begin{table}[b]
      \caption{Sample floating table}
      \label{labelforgloatingtable}
      \begin{tabular}{@{}lll@{}}
        \toprule
        header1 & header2 & header3 \\
        \midrule
        first row  & second column & last column \\
        second row & second column & last column \\
        \bottomrule
      \end{tabular}
    \end{table}

  \section{Equations}
    Using and referencing equations 
    is given with standard \LaTeX{} macros.
    \begin{equation}
      \frac{1}{m\cdot c^2} = \frac{1}{E}
      \label{equationlabel}
    \end{equation}
    This equation will be numbered properly and you
    can refer to them with appropriate
    \texttt{label} and \texttt{ref} command.
    Just like here in Eq.~\ref{equationlabel}.


  \section{Closing remarks}
    The Author(s) are kindly requested to submit, with the manuscript, 
    the names, and e-mail addresses of five potential reviewers. Note 
    that the editor retains the sole right to decide whether or not the 
    suggested reviewers are used. The names of potential reviewers should 
    be entered in the section Comments for the Editor in the 1st step of 
    online submission process.

    The editors reserve the right to modify 
    your source files to adapt it to journal
    requirements.

    \acknowledgement{
      If you would like to use any acknowledgement 
      please insert it into
      \texttt{acknowledgement} command.
    }

    References 

    \authornote{%
      Professor John Smith \email{johnsmith@xxx.com}\\
      PhD Jerry White \email{jerry@yyy.pl}
    }
    

%% this last part is just inclusion of source file 
%% only for documentation purposes
\clearpage
  \section*{Appendix}
  \subsection*{Listing of \texttt{sample\_paper.tex} file (source of this document)}
    \scriptsize
    \verbatiminput{./sample_article.tex}

\end{document}

The figures and subscriptions should be numbered below the figures and centred, 
e.g. >"Fig. 1. Text"< (align to the left of figure, use the font size of 11 
  points). 

Fig. 1. Title of figure

The tables should be numbered on the top of the table and centred, e.g. >"Table 1. Text< (align to the left of table, use the font size of 11 points).

Table 1. Title of table

